El presente proyecto se justifica por la necesidad de abordar los desafíos críticos que enfrenta la orientación vocacional en América Latina, buscando promover un acceso equitativo a la educación superior y reducir significativamente la deserción estudiantil. La implementación de un sistema inteligente de orientación vocacional se presenta como una solución integral que ofrece múltiples beneficios. En primer lugar, este sistema proporcionará orientación personalizada a cada estudiante, utilizando algoritmos de inteligencia artificial para analizar sus datos, identificar sus intereses, aptitudes y valores, y ofrecer recomendaciones de carrera que aumenten la probabilidad de éxito y satisfacción profesional. Además, el sistema garantizará el acceso a información actualizada y relevante sobre carreras, planes de estudio y perspectivas laborales, considerando las tendencias del mercado y las demandas de las industrias emergentes, lo que permitirá a los estudiantes tomar decisiones informadas y elegir carreras con alta demanda y potencial de crecimiento. La accesibilidad y escalabilidad del sistema, diseñado como un servicio para la multiplataforma de orientación vocacional, al ser accesible desde diversos dispositivos y ubicaciones, asegurará que un mayor número de estudiantes, especialmente aquellos en áreas rurales o con acceso limitado a recursos educativos, puedan beneficiarse de esta herramienta. Al proporcionar una orientación vocacional precisa y personalizada, el sistema reducirá la incertidumbre y la insatisfacción de los estudiantes, disminuyendo la probabilidad de cambios de carrera y abandono de estudios. Asimismo, este proyecto promoverá la equidad educativa, democratizando el acceso a la educación superior y brindando a todos los estudiantes, independientemente de su origen socioeconómico, la oportunidad de tomar decisiones informadas sobre su futuro profesional. Finalmente, al orientar a los estudiantes hacia carreras con alta demanda y potencial de crecimiento, el sistema contribuirá al desarrollo del talento humano en áreas estratégicas para el desarrollo económico y social de América Latina.