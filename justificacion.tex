El presente proyecto aborda la necesidad de implementar soluciones tecnológicas avanzadas en la orientación educativa, utilizando sistemas inteligentes para la evaluación y recomendación, como los descritos por Chauhan et al. (2020). La propuesta de un sistema de clasificación y recomendación para carreras universitarias se alinea con la creciente demanda de herramientas personalizadas que apoyen a los estudiantes en sus decisiones formativas.

La infraestructura tecnológica subyacente es fundamental. El diseño de sistemas basados en inteligencia artificial para entornos educativos requiere arquitecturas robustas que permitan procesar datos y ofrecer recomendaciones de manera eficiente y escalable (Chauhan et al., 2020). Además, la adopción efectiva de estas herramientas en entornos educativos depende de una cuidadosa orquestación y comunicación entre las partes interesadas, asegurando que la analítica del aprendizaje sea comprendida y utilizada adecuadamente (Prieto-Alvarez et al., 2018).

La relevancia de este enfoque se acentúa ante la transformación del mercado laboral. El Foro Económico Mundial (2023) destaca cómo las habilidades requeridas y las profesiones están en constante evolución, haciendo indispensable una orientación que prepare a los estudiantes para el futuro del trabajo. Un sistema que no solo recomiende carreras, sino que también sugiera áreas de especialización y genere itinerarios de estudio adaptados, responde directamente a esta necesidad de desarrollo continuo de competencias.

El contexto digital actual de los jóvenes en regiones como América Latina y el Caribe (UNICEF, 2020) justifica la implementación de una solución tecnológica accesible y adaptada a sus patrones de interacción con la información.

Finalmente, la ética y la seguridad son primordiales. Al manejar datos personales y académicos, el sistema debe adherirse estrictamente a las directrices sobre inteligencia artificial y educación, garantizando la protección de datos y la privacidad, tal como lo recomienda la UNESCO (2021). La construcción de confianza a través de prácticas responsables es esencial para la aceptación y el éxito del sistema.

En resumen, este proyecto integra consideraciones técnicas, educativas y éticas para ofrecer una herramienta de orientación vocacional inteligente, pertinente para las demandas actuales y futuras, y respetuosa con la privacidad del usuario.