A continuación se presenta el cronograma semanal de actividades para el desarrollo del servicio backend e infraestructura del sistema de orientación vocacional, con un enfoque metodológico claro que incluye fases de planeación, implementación, pruebas, análisis y documentación. El rango cubre desde el 1 de junio hasta el 30 de noviembre de 2025.

\begin{itemize}
  \item \textbf{Semana 1 (01-07 Jun)}: Planificación general del proyecto, delimitación de objetivos técnicos del backend y definición de herramientas de desarrollo.
  \item \textbf{Semana 2 (08-14 Jun)}: Diseño de la arquitectura general del servicio backend e infraestructura en la nube (servicios, bases de datos, autenticación, seguridad).
  \item \textbf{Semana 3 (15-21 Jun)}: Revisión bibliográfica complementaria sobre orientación vocacional y tecnologías para recolección de datos en servicios web.
  \item \textbf{Semana 4 (22-28 Jun)}: Desarrollo inicial del esquema de base de datos y endpoints de autenticación segura (login, registro, tokens).
  \item \textbf{Semana 5 (29 Jun-05 Jul)}: Implementación de sistema de almacenamiento seguro de datos personales y vocacionales.
  \item \textbf{Semana 6 (06-12 Jul)}: Construcción del módulo de clasificación para sugerencias de licenciaturas basado en datos del usuario.
  \item \textbf{Semana 7 (13-19 Jul)}: Integración de lógica de negocio para generación de temarios de estudio personalizados.
  \item \textbf{Semana 8 (20-26 Jul)}: Pruebas unitarias, control de errores y revisión de rendimiento inicial.
  \item \textbf{Semana 9 (27 Jul-02 Ago)}: Desarrollo del sistema de registro y edición de habilidades e intereses del usuario.
  \item \textbf{Semana 10 (03-09 Ago)}: Implementación de módulo de consulta de sugerencias y resultados por parte del usuario.
  \item \textbf{Semana 11 (10-16 Ago)}: Pruebas funcionales del sistema completo y revisión de seguridad. Simulación de carga.
  \item \textbf{Semana 12 (17-23 Ago)}: Recolección de datos reales o simulados para entrenamiento y validación del sistema.
  \item \textbf{Semana 13 (24-30 Ago)}: Análisis de resultados obtenidos, evaluación de precisión de las sugerencias, revisión crítica.
  \item \textbf{Semana 14 (31 Ago-06 Sep)}: Ajustes en la clasificación y resultados del sistema. Mejoras en la infraestructura de respuesta.
  \item \textbf{Semana 15 (07-13 Sep)}: Optimización de velocidad de respuesta e implementación de cachés.
  \item \textbf{Semana 16 (14-20 Sep)}: Pruebas de escalabilidad: simulaciones de usuarios concurrentes.
  \item \textbf{Semana 17 (21-27 Sep)}: Mejora de accesibilidad (compatibilidad móvil, baja conectividad).
  \item \textbf{Semana 18 (28 Sep-04 Oct)}: Validación cruzada de la integridad de datos, seguridad y experiencia de usuario.
  \item \textbf{Semana 19 (05-11 Oct)}: Redacción del informe técnico sobre la arquitectura y funcionamiento del backend.
  \item \textbf{Semana 20 (12-18 Oct)}: Elaboración del informe final de metodología, resultados y recomendaciones.
  \item \textbf{Semana 21 (19-25 Oct)}: Revisión del borrador del trabajo de graduación e integración de observaciones.
  \item \textbf{Semana 22 (26 Oct-01 Nov)}: Revisión final de resultados y comparación con objetivos.
  \item \textbf{Semana 23 (02-08 Nov)}: Redacción final de conclusiones y propuestas de mejora.
  \item \textbf{Semana 24 (09-15 Nov)}: Preparación de presentación de resultados y defensa.
  \item \textbf{Semana 25 (16-22 Nov)}: Simulacros de presentación y revisión del discurso.
  \item \textbf{Semana 26 (23-30 Nov)}: Entrega final del informe y documentación técnica completa.
\end{itemize}
