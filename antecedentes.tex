A continuación se presenta el cronograma semanal de actividades para el desarrollo del servicio backend e infraestructura del sistema de orientación vocacional, con un enfoque metodológico, los cuales comprende fases de planeación, diseño, implementación, pruebas, despliegue y documentación. El rango cubre desde el 1 de junio hasta el 30 de noviembre de 2025.

\begin{itemize}
  \item \textbf{Semana 1 (01-07 Jun)}: Planificación general del proyecto, definición de objetivos técnicos y herramientas de desarrollo.
  \item \textbf{Semana 2 (08-14 Jun)}: Diseño de la arquitectura backend e infraestructura en la nube (servicios, autenticación, bases de datos).
  \item \textbf{Semana 3 (15-21 Jun)}: Definición del modelo de seguridad, autenticación y autorización (JWT, OAuth).
  \item \textbf{Semana 4 (22-28 Jun)}: Desarrollo de endpoints de autenticación segura (registro, login, tokens).
  \item \textbf{Semana 5 (29 Jun-05 Jul)}: Implementación de almacenamiento seguro de preferencias académicas y configuraciones.
  \item \textbf{Semana 6 (06-12 Jul)}: Integración con modelos de clasificación y recomendación de carreras.
  \item \textbf{Semana 7 (13-19 Jul)}: Desarrollo del módulo de generación de temarios personalizados.
  \item \textbf{Semana 8 (20-26 Jul)}: Implementación de sistema de cacheo y optimización de rendimiento.
  \item \textbf{Semana 9 (27 Jul-02 Ago)}: Desarrollo de sistema de logging estructurado y trazabilidad.
  \item \textbf{Semana 10 (03-09 Ago)}: Desarrollo de endpoints para la consulta de resultados y seguimiento por parte del usuario.
  \item \textbf{Semana 11 (10-16 Ago)}: Pruebas funcionales del sistema, control de errores y validación de seguridad.
  \item \textbf{Semana 12 (17-23 Ago)}: Pruebas de carga y estrés (simulación de usuarios concurrentes).
  \item \textbf{Semana 13 (24-30 Ago)}: Optimización de velocidad de respuesta y tiempos de procesamiento.
  \item \textbf{Semana 14 (31 Ago-06 Sep)}: Implementación de balanceo de carga y autoescalado.
  \item \textbf{Semana 15 (07-13 Sep)}: Configuración de monitoreo y alertas con herramientas de observabilidad.
  \item \textbf{Semana 16 (14-20 Sep)}: Pruebas de recuperación ante fallos y validación de alta disponibilidad.
  \item \textbf{Semana 17 (21-27 Sep)}: Revisión de accesibilidad y compatibilidad multiplataforma.
  \item \textbf{Semana 18 (28 Sep-04 Oct)}: Auditoría final del sistema y revisión de políticas de seguridad.
  \item \textbf{Semana 19 (05-11 Oct)}: Redacción del informe técnico sobre arquitectura y funcionamiento del sistema.
  \item \textbf{Semana 20 (12-18 Oct)}: Elaboración del informe final de resultados y metodología.
  \item \textbf{Semana 21 (19-25 Oct)}: Revisión del borrador del trabajo de graduación.
  \item \textbf{Semana 22 (26 Oct-01 Nov)}: Integración de observaciones y revisión final.
  \item \textbf{Semana 23 (02-08 Nov)}: Redacción de conclusiones y propuestas de mejora técnica.
  \item \textbf{Semana 24 (09-15 Nov)}: Preparación de la defensa y presentación del proyecto.
  \item \textbf{Semana 25 (16-22 Nov)}: Simulacros de presentación.
  \item \textbf{Semana 26 (23-30 Nov)}: Entrega final del informe y documentación técnica completa.
\end{itemize}
