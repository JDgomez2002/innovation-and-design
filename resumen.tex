El presente trabajo aborda el desarrollo de un servicio backend e infraestructura para un sistema inteligente de orientación vocacional dirigido a estudiantes graduandos con aspiración a cursar estudios universitarios en América Latina. La investigación responde a la necesidad crítica de mejorar los procesos de elección de carrera en la región, donde factores como la desigualdad económica, la falta de información actualizada y la rápida evolución del mercado laboral dificultan la toma de decisiones informadas por parte de los estudiantes.

El proyecto integra tecnologías avanzadas de inteligencia artificial y análisis de datos para implementar un sistema de clasificación y recomendación que, basado en las preferencias académicas, fortalezas intelectuales y gustos personales de los usuarios, genera sugerencias personalizadas de licenciaturas, temarios de estudio a medida y recomendaciones de especialización profesional. La arquitectura desarrollada prioriza aspectos fundamentales como la seguridad de la información, la velocidad de respuesta, la escalabilidad para atender múltiples usuarios simultáneamente y la accesibilidad desde diversos contextos geográficos y tecnológicos.

La metodología implementada combina enfoques cuantitativos y cualitativos, incluyendo revisión documental sobre orientación vocacional, instrumentación psicométrica, recolección y análisis de datos, así como la identificación de problemas educacionales específicos de la región latinoamericana. El desarrollo técnico sigue un cronograma estructurado que abarca desde la planificación y diseño arquitectónico hasta la implementación, pruebas, optimización y validación del sistema.

Los resultados obtenidos demuestran que la integración de tecnologías de inteligencia artificial en los procesos de orientación vocacional ofrece ventajas significativas en términos de personalización, accesibilidad y pertinencia de las recomendaciones. El sistema desarrollado constituye una herramienta valiosa para apoyar a los estudiantes en la crucial tarea de elegir una carrera universitaria alineada con sus capacidades e intereses, contribuyendo así a reducir la deserción académica y a mejorar la satisfacción profesional a largo plazo.

Este trabajo representa una contribución significativa al campo de la orientación vocacional en América Latina, proponiendo un enfoque tecnológico innovador que responde a las particularidades socioeconómicas y educativas de la región, con potencial para impactar positivamente en las trayectorias académicas y profesionales de los estudiantes.

\textbf{Palabras clave:} orientación vocacional, inteligencia artificial, backend, infraestructura cloud, sistemas de recomendación, educación superior, América Latina.

