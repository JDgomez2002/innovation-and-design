Con base en la planificación y el desarrollo preliminar del protocolo para el servicio backend e infraestructura del sistema inteligente de orientación vocacional, se proponen las siguientes recomendaciones para la ejecución del proyecto y futuras investigaciones:
\begin{enumerate}
\item \textbf{Consideración de un modelo de datos expansible:} Se recomienda diseñar el sistema con la capacidad de incorporar fácilmente nueva información sobre tendencias laborales emergentes y competencias específicas requeridas en el mercado latinoamericano. Esta flexibilidad permitirá mejorar continuamente la precisión y relevancia de las recomendaciones que generará el sistema.
\item \textbf{Planificación simultánea de la interfaz de usuario:} Aunque el enfoque actual se centra en el backend e infraestructura, se sugiere considerar desde el diseño inicial los requerimientos para futuras interfaces de usuario multiplataforma (web, móvil, PWA) que aprovecharán las capacidades del backend, priorizando la experiencia de usuario y la accesibilidad para diversos contextos tecnológicos presentes en la región.
\item \textbf{Previsión de mecanismos de integración:} Se recomienda incluir en el diseño del backend APIs y conectores que faciliten la futura integración con plataformas educativas ya establecidas (LMS, sistemas de gestión escolar, portales universitarios) para ampliar el alcance del sistema y enriquecer la calidad de los datos disponibles para el análisis vocacional.
\item \textbf{Arquitectura preparada para técnicas avanzadas de ML:} El sistema debería diseñarse considerando la futura incorporación de técnicas de aprendizaje profundo y procesamiento de lenguaje natural que permitan analizar información no estructurada (ensayos personales, descripciones de intereses) para enriquecer los perfiles vocacionales de los usuarios.
\item \textbf{Incorporación de mecanismos para retroalimentación:} Se recomienda incluir desde el diseño inicial del backend la capacidad de recopilar sistemáticamente información sobre la satisfacción de los usuarios con las recomendaciones recibidas, previendo la creación de un ciclo de mejora continua para los futuros algoritmos de recomendación.
\item \textbf{Establecimiento temprano de colaboraciones institucionales:} Se sugiere iniciar acercamientos con universidades y centros educativos durante la fase de desarrollo para preparar futuras validaciones de los modelos de recomendación, así como para facilitar la implementación piloto del sistema en contextos reales una vez completado.
\item \textbf{Diseño con conciencia energética:} Considerando el impacto ambiental de las soluciones tecnológicas, se recomienda incorporar en el diseño de la infraestructura estrategias de optimización energética en la nube, como la programación de recursos basada en la demanda o la selección de proveedores con compromiso de energías renovables.
\item \textbf{Previsión de componentes de explicabilidad:} Se sugiere considerar desde el diseño inicial la incorporación de capacidades de IA explicable (XAI) que permitirán a los usuarios comprender las razones detrás de las recomendaciones recibidas, fomentando así la transparencia y la confianza en el futuro sistema.
\item \textbf{Consideración de aspectos multiculturales:} Se recomienda diseñar los modelos y algoritmos con la flexibilidad necesaria para considerar en el futuro las diferencias culturales y lingüísticas específicas de los diversos países latinoamericanos, mejorando así la pertinencia contextual de las recomendaciones vocacionales.
\item \textbf{Planificación para posibles modelos híbridos:} Prever en la arquitectura la posibilidad de implementar sistemas híbridos que combinen la orientación automatizada con la intervención de consejeros humanos en momentos clave del proceso, maximizando los beneficios de ambos enfoques.
\item \textbf{Extensibilidad hacia otros niveles educativos:} Diseñar el sistema con la capacidad de adaptarse en el futuro a otros niveles educativos como educación media o posgrados, creando así la base para un potencial ecosistema integral de orientación a lo largo de la trayectoria formativa.
\item \textbf{Establecimiento de políticas de actualización de contenidos:} Incorporar en el diseño mecanismos que faciliten la actualización periódica de la información sobre carreras, temarios y especialidades, garantizando la futura vigencia y relevancia continua de las recomendaciones que proporcionará el sistema.
\end{enumerate}
La consideración de estas recomendaciones durante el desarrollo del proyecto podría potenciar significativamente el impacto positivo del sistema propuesto, contribuyendo a sentar las bases para una transformación digital de los procesos de orientación vocacional en América Latina y promoviendo en el futuro decisiones académicas más informadas y satisfactorias entre los estudiantes de la región.