\section{Bibliografía}

Chauhan, A., Jagannathan, S., \& Sikka, G. (2020). Artificial Intelligence-Based Student Learning Evaluation and Recommendation System. En Intelligent System Design (pp. 513-523). Springer Singapore. \url{https://link.springer.com/chapter/10.1007/978-981-15-5400-1_50}
\newline

Foro Económico Mundial. (2023). The Future of Jobs Report 2023. WEF. \url{https://www.weforum.org/publications/the-future-of-jobs-report-2023/}
\newline

Prieto-Alvarez, L., Martinez-Maldonado, R., Anderson, T., \& Buckingham Shum, S. (2018). Orchestrating learning analytics (OrLA): Supporting inter-stakeholder communication about adoption of learning analytics at the classroom level. Proceedings of the 8th International Conference on Learning Analytics and Knowledge (LAK '18). Association for Computing Machinery, New York, NY, USA, 446–450. \url{https://www.researchgate.net/publication/328840638_Orchestrating_learning_analytics_OrLA_Supporting_inter-stakeholder_communication_about_adoption_of_learning_analytics_at_the_classroom_level}
\newline

UNESCO. (2021). AI and education: Guidance for policy-makers. UNESCO Publishing. \url{https://unesdoc.unesco.org/ark:/48223/pf0000376709}
\newline

UNICEF. (2020). Niños, niñas y adolescentes en América Latina y el Caribe 2020. UNICEF LACRO. \url{https://www.unicef.org/lac/informes/ninos-y-ninas-en-america-latina-y-el-caribe-2020} (Nota: Este enlace dirige a una página resumen/portal del estado de la infancia en 2020 para la región.)