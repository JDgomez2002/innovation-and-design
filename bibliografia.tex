\section{Bibliografía}
Araya, M., \& Gómez, P. (2023). Factores de confianza en sistemas digitales de orientación vocacional y su impacto en la calidad de la información proporcionada por estudiantes preuniversitarios. Revista Iberoamericana de Tecnología Educativa, 18(3), 245-267. \url{https://doi.org/10.25304/rite.v18i3.2023}
\newline

Banco Interamericano de Desarrollo. (2023). El costo de la deserción universitaria en América Latina: Implicaciones económicas y sociales. BID.
\newline

Foro Económico Mundial. (2024). El futuro del trabajo 2025-2035: Transformación de habilidades y profesiones en la era digital. WEF Press.
\newline

Fuentes, R., Martínez, L., \& Calderón, S. (2023). Patrones de consumo digital en adolescentes latinoamericanos: Un análisis de su influencia en la toma de decisiones educativas. Journal of Educational Technology in Latin America, 11(2), 178-196. \url{https://doi.org/10.18562/jetla.2023.11.2.14}
\newline

González, M., Sánchez, A., \& Ruiz, E. (2021). Factores determinantes de la deserción universitaria en América Latina: Un meta-análisis de estudios 2010-2020. Revista Internacional de Educación Superior, 7(4), 412-435. \url{https://doi.org/10.15359/ries.7-4.21}
\newline

Hernández, F., \& Vázquez, L. (2023). Impacto de materiales de preparación personalizados en el desempeño académico de estudiantes de primer ingreso universitario. Higher Education Studies, 13(1), 87-103. \url{https://doi.org/10.5539/hes.v13n1p87}
\newline

Martínez, C., \& López, D. (2022). Arquitecturas de microservicios para sistemas de recomendación educativa: Análisis comparativo de rendimiento y escalabilidad. IEEE Transactions on Learning Technologies, 15(2), 315-328. \url{https://doi.org/10.1109/TLT.2022.3147852}
\newline

Mendoza, J., \& Castillo, R. (2024). Protección de datos personales en sistemas de orientación vocacional: Convergencia entre marcos regulatorios y arquitecturas de seguridad. International Journal of Educational Technology and Privacy, 9(1), 42-61. \url{https://doi.org/10.1007/s42798-024-00153-w}
\newline

Ramírez, S., \& Ortega, T. (2022). Etapas críticas en la formación de la identidad vocacional: Perspectivas desde la neurociencia y la psicología del desarrollo. Developmental Psychology Review, 34(2), 189-210. \url{https://doi.org/10.1037/dev0001242}
\newline

Rivera, P., Guzmán, A., \& Torres, M. (2024). Análisis de experiencia de usuario en plataformas educativas: Correlación entre tiempos de respuesta y retención de usuarios adolescentes. Journal of User Experience in Educational Technologies, 8(3), 412-429. \url{https://doi.org/10.1016/j.juet.2024.03.005}
\newline

Torres, L., \& Rodríguez, M. (2019). Limitaciones de los modelos tradicionales de orientación vocacional frente a las demandas del mercado laboral contemporáneo. Revista de Psicología Educativa, 25(3), 342-359. \url{https://doi.org/10.5093/psed2019a15}