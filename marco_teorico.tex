La presente investigación adopta un enfoque de ingeniería tecnológica aplicada, orientado al diseño, implementación y validación de una infraestructura backend segura, escalable y de alto rendimiento, que sirva de soporte a un sistema inteligente de orientación vocacional. El enfoque metodológico sigue los principios del desarrollo ágil y DevOps, priorizando prácticas de diseño de sistemas distribuidos, seguridad informática, disponibilidad de servicios y eficiencia computacional.

\section{Análisis y planeación}

Se realiza una definición detallada de los requerimientos funcionales y no funcionales del servicio backend:

\begin{itemize}
    \item Seguridad de la información (autenticación JWT, cifrado en tránsito y en reposo).
    \item Escalabilidad horizontal y vertical.
    \item Alta disponibilidad y tolerancia a fallos.
    \item Tiempo de respuesta óptimo.
    \item Modularidad mediante microservicios.
    \item Integración con modelos de recomendación y clasificación externos.
\end{itemize}

\section{Diseño de la arquitectura backend}

\begin{itemize}
    \item Diseño de arquitectura basada en microservicios y APIs RESTful.
    \item Especificación de flujos de datos y control, conforme al diagrama de secuencia propuesto.
    \item Selección de tecnologías (Docker, Kubernetes, API Gateway, OAuth, etc.).
    \item Definición de políticas de autenticación, autorización, logging y monitoreo.
\end{itemize}

\section{Implementación del sistema}

\begin{itemize}
    \item Construcción de endpoints para autenticación, consulta de sugerencias, y resultados de modelos de IA.
    \item Integración con servicios de clasificación y generación de recomendaciones vocacionales.
    \item Implementación de base de datos segura y sistemas de caché para optimizar la latencia.
    \item Desarrollo de herramientas para trazabilidad, auditoría y control de errores.
\end{itemize}

\section{Pruebas y validación}

\begin{itemize}
    \item Pruebas unitarias y de integración.
    \item Pruebas de carga y estrés para validar la escalabilidad del sistema.
    \item Pruebas de seguridad contra amenazas comunes (XSS, CSRF, inyecciones).
    \item Simulación de concurrencia para evaluar rendimiento bajo alta demanda.
\end{itemize}

\section{Despliegue e infraestructura}

\begin{itemize}
    \item Automatización del proceso de integración y entrega continua (CI/CD).
    \item Despliegue en infraestructura cloud (AWS, GCP, o Azure).
    \item Implementación de balanceadores de carga, autoescalado y recuperación ante fallos.
    \item Monitoreo activo con herramientas como Prometheus, Grafana y alertas configuradas.
\end{itemize}

\section{Documentación y evaluación}

\begin{itemize}
    \item Documentación técnica del backend (Swagger/OpenAPI).
    \item Evaluación de métricas clave (latencia, disponibilidad, rendimiento).
    \item Análisis de riesgos y propuesta de mejoras para futuras versiones del sistema.
\end{itemize}
