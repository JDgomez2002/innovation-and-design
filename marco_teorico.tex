La presente investigación adopta un enfoque metodológico mixto, combinando el análisis cuantitativo mediante herramientas psicométricas y encuestas, y el análisis cualitativo mediante revisión documental. Este trabajo se orienta al área de la educación, abarcando tanto la educación privada como la pública, y está dirigido a resolver los problemas educacionales que afectan a América Latina. El objetivo es explorar la relación entre los rasgos de personalidad, las habilidades percibidas y la elección vocacional de los estudiantes universitarios, así como identificar patrones que expliquen los casos de deserción o cambio de carrera, sentando las bases para recomendaciones dirigidas a mejorar los procesos de orientación y apoyo en el ámbito educativo.

\section{Observación y búsqueda documental}

Se llevará a cabo una revisión exhaustiva de literatura y fuentes secundarias relevantes en psicología, orientación vocacional, y en el contexto educacional de la región latinoamericana. Esta fase permitirá fundamentar teóricamente la investigación y seleccionar herramientas válidas para la medición.

\begin{itemize}
    \item Revisión de teorías sobre personalidad, especialmente el modelo de los cinco grandes factores.
    \item Análisis de pruebas psicométricas aplicables a contextos vocacionales.
    \item Revisión de literatura sobre la correlación entre personalidad, habilidades y elección vocacional.
    \item Investigación de tendencias laborales actuales y futuras por carrera.
    \item Consulta de estudios previos sobre deserción universitaria y satisfacción estudiantil, con especial énfasis en el contexto latinoamericano.
\end{itemize}

\section{Instrumentación y recolección de datos}

Los datos se recopilarán mediante instrumentos psicométricos y encuestas estructuradas, aplicados a estudiantes universitarios de instituciones tanto públicas como privadas. Además, se integrarán estadísticas institucionales para complementar el análisis.

\begin{itemize}
    \item Aplicación del test de personalidad IPIP-NEO-120 de Johnson (2014), basado en el modelo NEO-PI-R de Costa y McCrae (1992), disponible en: \url{https://ipip.ori.org/30FacetNEO-PI-RItems.htm}.
    \item Encuestas a estudiantes para recopilar datos sobre:
    \begin{itemize}
        \item Habilidades percibidas.
        \item Grado de satisfacción con la carrera actual.
        \item Motivaciones de elección vocacional.
        \item Antecedentes de cambio de carrera o intención de hacerlo.
        \item Percepción del acompañamiento institucional.
    \end{itemize}
    \item Recolección de datos institucionales:
    \begin{itemize}
        \item Tasas de deserción por carrera.
        \item Registros de cambio de carrera.
        \item Resultados de encuestas de satisfacción estudiantil.
        \item Datos demográficos generales.
    \end{itemize}
\end{itemize}

\section{Registro de datos y creación del dataset}

Una vez recolectada la información, ésta será sistematizada y almacenada en un dataset estructurado, respetando principios de confidencialidad y calidad de datos.

\begin{itemize}
    \item Definición de variables cuantitativas y cualitativas.
    \item Codificación de datos para análisis posterior.
    \item Anonimización de información personal.
    \item Almacenamiento del dataset en software de análisis estadístico (Python y almacenar resultados en formato CSV).
\end{itemize}

\section{Análisis e interpretación de datos}

El dataset será analizado con técnicas estadísticas y de correlación para identificar patrones relevantes en la elección vocacional, satisfacción y deserción.

\begin{itemize}
    \item Cálculo de correlaciones entre rasgos de personalidad y satisfacción vocacional.
    \item Identificación de perfiles frecuentes en casos de cambio de carrera.
    \item Análisis predictivo de deserción basado en personalidad y habilidades.
    \item Segmentación de la muestra para identificar diferencias significativas.
\end{itemize}

\section{Identificación de problemas}

Con base en los datos analizados, se buscará detectar los principales factores que dificultan la elección vocacional adecuada y la permanencia en el sistema educativo.

\begin{itemize}
    \item Falta de información o autoconocimiento al elegir una carrera.
    \item Desajuste entre la personalidad del estudiante y el perfil de la carrera.
    \item Frustración o desmotivación vinculada a la elección educativa.
    \item Necesidad de una orientación vocacional más personalizada y adaptada, considerando los desafíos tanto de la educación pública como privada.
\end{itemize}

\section{Recomendaciones}

Los resultados del estudio permitirán emitir recomendaciones concretas, orientadas a transformar y mejorar los procesos de orientación vocacional en instituciones educativas, con miras a resolver los problemas que afectan a la educación en América Latina.

\begin{itemize}
    \item Integrar herramientas psicométricas en los procesos de orientación desde el ingreso a las instituciones.
    \item Fortalecer el acompañamiento institucional mediante asesorías personalizadas en el ámbito educativo.
    \item Prevenir la deserción mediante estrategias de apoyo individualizadas y seguimiento académico.
    \item Desarrollar plataformas inteligentes que sugieran carreras y especializaciones basadas en la personalidad, habilidades e intereses del estudiante.
    \item Orientar las recomendaciones hacia la mejora del sistema educativo, contribuyendo al diseño de políticas y estrategias que beneficien tanto a la educación pública como privada en América Latina, tomando en cuenta las referencias y citas bibliográficas disponibles en la literatura (e.g., (Chauhan et al., 2020), (Prieto-Alvarez et al., 2018), (UNICEF, 2020)).
\end{itemize}