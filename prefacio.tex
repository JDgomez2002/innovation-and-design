En el transcurso de mi formación académica, he sido testigo de la compleja travesía que significa para muchos estudiantes latinoamericanos la elección de una carrera universitaria. Esta decisión, que marca profundamente el futuro profesional y personal, frecuentemente se toma con información limitada y sin las herramientas adecuadas para un análisis introspectivo efectivo.
El presente trabajo surge de la convergencia entre mi pasión por las tecnologías de la información y mi interés por contribuir a resolver problemáticas educativas en nuestra región. La orientación vocacional, tradicionalmente relegada a métodos convencionales y generalistas, representa un campo propicio para la aplicación de tecnologías avanzadas como la inteligencia artificial y el análisis de datos, ofreciendo así soluciones personalizadas y accesibles.
Este proyecto no habría sido posible sin el apoyo invaluable de mi asesor, el Ing. Ludwing Cano, cuya guía y conocimiento han sido fundamentales para navegar los aspectos técnicos y metodológicos de esta investigación. Asimismo, extiendo mi agradecimiento a la Facultad de Ingeniería de la Universidad del Valle de Guatemala por proporcionar el entorno académico que ha estimulado mi desarrollo profesional y ha fomentado mi interés por la aplicación de soluciones tecnológicas a problemas sociales relevantes.
Agradezco también a mis compañeros de estudio, cuyas perspectivas y colaboración enriquecieron significativamente este trabajo. Sus aportes durante las sesiones de discusión y retroalimentación han sido invaluables para refinar los conceptos y metodologías aplicados en este proyecto.
Finalmente, dedico este trabajo a mi familia, cuyo apoyo incondicional ha sido el pilar fundamental durante toda mi formación académica. Su comprensión, paciencia y aliento constante han sido esenciales para culminar con éxito esta etapa de mi vida profesional.
Confío en que este proyecto contribuirá de manera significativa al campo de la orientación vocacional en América Latina, aprovechando el potencial de las tecnologías emergentes para democratizar el acceso a herramientas de orientación vocacional de calidad, adaptadas a las necesidades específicas de nuestra región.
\newline
\newline
José Daniel Gómez Cabrera
\newline
\newline
Guatemala, Mayo del 2025