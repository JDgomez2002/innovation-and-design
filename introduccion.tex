La elección de una carrera universitaria representa un momento crucial en la vida de cualquier estudiante, marcando el inicio de su trayectoria profesional y personal. Sin embargo, este proceso de decisión puede resultar abrumador, especialmente en el contexto latinoamericano, donde el acceso a la educación superior privada es limitado y muchos estudiantes enfrentan incertidumbre sobre sus aptitudes e intereses. La falta de orientación vocacional adecuada a menudo conduce a elecciones desacertadas, resultando en cambios de carrera, abandono de estudios y frustración profesional.

En América Latina, la educación superior se enfrenta a desafíos significativos. La desigualdad económica limita el acceso a instituciones privadas de calidad, mientras que las universidades públicas a menudo carecen de recursos para brindar una orientación vocacional personalizada y actualizada. Esta situación se agrava por la falta de información sobre las diversas opciones de carrera y los requisitos académicos, lo que dificulta la toma de decisiones informadas.

Además, la globalización y la rápida evolución del mercado laboral exigen profesionales con habilidades especializadas y adaptabilidad. Los estudiantes necesitan comprender las tendencias del mercado, las demandas de las industrias emergentes y las competencias necesarias para sobresalir en sus campos. Sin embargo, la orientación vocacional tradicional a menudo se centra en pruebas estandarizadas y evaluaciones genéricas, sin considerar las particularidades de cada estudiante y las oportunidades laborales en su entorno.

Ante este panorama, el desarrollo de un sistema inteligente de orientación vocacional se presenta como una solución innovadora y necesaria. Este sistema, basado en tecnologías de inteligencia artificial y análisis de datos, puede proporcionar a los estudiantes una orientación personalizada y precisa, considerando sus intereses, aptitudes, valores y contexto socioeconómico. Al ofrecer información detallada sobre las carreras, los planes de estudio y las perspectivas laborales, el sistema puede empoderar a los estudiantes para tomar decisiones informadas y construir un futuro profesional exitoso.

El presente trabajo propone el desarrollo de un servicio y su infraestructura para abastecer un sistema inteligente de orientación vocacional, generación de temarios de estudio y sugerencias de especializaciones a nivel profesional, para estudiantes graduandos con aspiración a estudiar una licenciatura. Estos serán generados y clasificados en función de sus preferencias académicas, fortalezas intelectuales y gustos personales. Estos datos se generarán por medio de un sistema de clasificación y generación, este se implementará basado en algoritmos de aprendizaje automático y análisis de datos, para generar las sugerencias personalizadas de licenciaturas, temarios de estudio y sugerencias de especializaciones a nivel profesional. El servicio estará diseñado para garantizar la seguridad de la información, la velocidad de respuesta, la exactitud de la información sugerida, y la escalabilidad de peticiones y recursos, necesarias para atender a un gran número de usuarios de manera simultánea, sin comprometer la calidad de la información ni la experiencia del usuario.
